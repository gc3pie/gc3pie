\documentclass[english,serif,mathserif,xcolor=pdftex,dvipsnames,table]{beamer}
\usepackage{gc3}

\title[The Session Based Script]{%
GC3Pie - The Session Based Script
}
\author[Antonio Messina]{%
  GC3: Grid Computing Competence Center, \\
  University of Zurich
}


\date{Oct.~1, 2012}

\begin{document}

\maketitle

\begin{frame}
  \frametitle{Exercise 3.C} 

  Update the exercice 3.B so that it runs 10 copies of the
  \textbf{CpuinfoApplication} and collect statistics about CPU models
  (how many unique ``model name'' strings)
\end{frame}

\begin{frame}[fragile]
  \frametitle{Why exercise 3.C failed?}
  \begin{itemize}
  \item \textbf{localhost} resource does not allow more than two jobs
    at the same time.

    \pause
    \begin{itemize}
    \item[$\Rightarrow$] \textbf{Engine}, a more advanced
      \textit{version} of \textbf{Core}, is able to manage a list of
      jobs and to deal with these situations.
    \end{itemize}
    
    \pause
    \+
  \item if the script is killed, all information about the jobs are
    lost.
    \pause
    \begin{itemize}
    \item [$\Rightarrow$] a \textbf{Session} is a \textit{persistent
        collection of jobs}. They are saved on the filesystem or a DB.
    \end{itemize}
    
    \pause  
    \+
  \item some logic is common to any script, including code to
    \textit{glue} all together and to parse command line options.  \pause
    \begin{itemize}
    \item [$\Rightarrow$] a \textbf{SessionBasedScript} automatically
      create an \textbf{Engine}, group all the jobs into a
      \textbf{Session}, accept some commonly used options and much more.
    \end{itemize}
  \end{itemize}
\end{frame}


\begin{frame}[fragile]
  \frametitle{Creating a SessionBasedScript is easy}
Create a file named \texttt{demoscript.py}:
\+
  \begin{lstlisting}
from gc3libs import Application
from gc3libs.cmdline import SessionBasedScript

class Gdemo(SessionBasedScript):
    """ Gdemo script """
    version='1.0'

    def new_tasks(self, extra):
        return [
            Application(['/bin/hostname'], [], [],
                        stdout='stdout.txt', **extra),
               ]

if __name__ == "__main__":
    from demo import Gdemo
    Gdemo().run()
  \end{lstlisting}
\end{frame}

\begin{frame}[fragile]
  \frametitle{Creating a SessionBasedScript is easy}
  \begin{lstlisting}
from gc3libs import Application
@\HL{from gc3libs.cmdline import SessionBasedScript}@

class Gdemo(@\HL{SessionBasedScript}@):
    """ Gdemo script """
    version='1.0'

    def new_tasks(self, extra):
        return [
            Application(['/bin/hostname'], [], [],
                        stdout='stdout.txt', **extra),
               ]

if __name__ == "__main__":
    from demo import Gdemo
    Gdemo().run()
  \end{lstlisting}
\end{frame}


\begin{frame}[fragile]
  \frametitle{Creating a SessionBasedScript is easy}
  \begin{lstlisting}
from gc3libs import Application
from gc3libs.cmdline import SessionBasedScript

class Gdemo(SessionBasedScript):
    """ Gdemo script """
    version='1.0'

    @\HL{def new\_tasks(self, extra):}@
        return [
            Application(['/bin/hostname'], [], [],
                        stdout='stdout.txt', **extra),
               ]

if __name__ == "__main__":
    from demo import Gdemo
    Gdemo().run()
  \end{lstlisting}
\end{frame}

\begin{frame}[fragile]
  \frametitle{Creating a SessionBasedScript is easy}
  \begin{lstlisting}
from gc3libs import Application
from gc3libs.cmdline import SessionBasedScript

class Gdemo(SessionBasedScript):
    """ Gdemo script """
    version='1.0'

    def new_tasks(self, @\HL{extra}@):
        return [
            Application(['/bin/hostname'], [], [],
                        stdout='stdout.txt', @\HL{**extra}@),
               ]

if __name__ == "__main__":
    from demo import Gdemo
    Gdemo().run()
  \end{lstlisting}
\end{frame}

\begin{frame}[fragile]
  \frametitle{Running the script}

  \begin{lstlisting}[language=sh]
kenny:~$ python demoscript.py -C 1    
[...]
        NEW   0/1    (0.0%)  
    RUNNING   0/1    (0.0%)  
    STOPPED   0/1    (0.0%)  
  SUBMITTED   0/1    (0.0%)  
 TERMINATED   1/1   (100.0%) 
TERMINATING   0/1    (0.0%)  
    UNKNOWN   0/1    (0.0%)  
         ok   1/1   (100.0%) 
      total   1/1   (100.0%) 
  \end{lstlisting}%$
  \pause
  \+

  In the current directory you will find two directories:
  \begin{description}
  \item[demoscript] the directory containing the session data.
  \item[Application-N1] the directory containing the output of the
    application.
  \end{description}
\end{frame}

\begin{frame}[fragile]
  \frametitle{The session directory}

  \begin{itemize}
  \item It contains internal data used by gc3pie.
  \item You can specify a different name using the option \lstinline|-s SESSION_NAME|
  \item If a session already exists, the script will \textbf{not} create
    new jobs, but instead, will update the status of the jobs in the
    current session.
  \end{itemize}

  The bottom line is\ldots 

  \pause 
  \begin{center}
    \textit{\textbf{don't touch it!}}
  \end{center}
  
\end{frame}


\begin{frame}[fragile]
  \frametitle{The output directory}
  \begin{itemize}
  \item If you don't specify an output directory for your job, the
    \textbf{SessionBasedScript} class will do it for you.
  \item If an output directory already exists, this will be
    \textit{renamed} and never overwritten.
  \item If you pass the option \lstinline|-o DIRECTORY| to the script,
    all the output dirs will be saved inside that directory
  \end{itemize}

  In our case, the output directory \lstinline|Application-N1| will
  contain a file \lstinline|stdout.txt| with the output of the
  application.

  \pause\+
  \centering\textit{Don't trust me, check yourself}

\end{frame}

\begin{frame}
  \frametitle{Session Based Script - command line options}
  \begin{description}
  \item[--help] show an help message and exits
  \item[-C NUM] Keep running, monitoring jobs and possibly submitting
    new ones or fetching results every NUM seconds. Exit when all jobs
    are finished.
  \item[-o DIR] Output files from all jobs will be collected in the
    specified DIRECTORY path.
  \item[-s PATH] Store the session information in the directory at
    PATH
  \item[-r NAME] Submit jobs to a specific computational resources.
    NAME is a resource name or comma-separated list of such names. 
  \item[-J NUM] Set the max NUMber of jobs (default: 50) running at
    the same time.

  \end{description}
\end{frame}


\begin{frame}[fragile]
  \frametitle{Passing requirements to the application}
  Some options are used to specify some requirements of the applications:
  \begin{description}
  \item[-c NUM] Set the number of CPU cores required for each job.
  \item[-m GB] Set the amount of memory required per execution core
  \item[-w DURATION] Set the time limit for each job; default is 8
    hours.
  \end{description}

  \pause
  and are automatically passed to the application, if you remember to
  do it!
  \begin{lstlisting}
    def new_tasks(self, @\HL{extra}@):
        return [
            Application(['/bin/hostname'], [], [],
                        stdout='stdout.txt', @\HL{**extra}@),
               ]
  \end{lstlisting}
\end{frame}

\begin{frame}[fragile]
  \frametitle{Exercise 5.A}
  \begin{itemize}
  \item Create a \textbf{GHelloWorld} application that writes the
    string \texttt{Hello, World!} into a file.
  \item Create a \textbf{GHelloScript} script that runs 20 instances of
    the \textbf{GHelloWorld} application.
  \end{itemize}
\end{frame}

\begin{frame}[fragile]
  \frametitle{How to add command line options}
  To setup new arguments you must override the \lstinline|setup_options|
  method of the script.

  \begin{lstlisting}[showstringspaces=false]
def setup_options(self):
    self.add_param('-x', '--option', dest='varname',
                   default="Default value",
                   help="Meaningful help string"
                   "which will be printed"
                   "by the --help option")    
  \end{lstlisting}

  \begin{itemize}
  \item Supports short and/or long options.
  \item if \lstinline|dest='varname'| then the content will be
    available inside the script as \lstinline|self.params.varname|
  \end{itemize}

  \begin{references}
    \url{http://docs.python.org/dev/library/argparse.html#argparse.ArgumentParser.add_argument}
  \end{references}
\end{frame}




\begin{frame}[fragile]
  \frametitle{Exercise 5.B}
  Starting from the  Exercise 5.A:
  \begin{itemize}
  \item add an option \lstinline|--string| which accept a string
    argument, which is the string that will be
  printed by the application instead of \texttt{Hello, World!}
\item add an option \lstinline|--copies| that accept an integer
  argument (by default, 1), and modify \lstinline|new_tasks| so that
  it will run \lstinline|copies| number of the \textbf{GHelloWorld} application.
  
  \end{itemize}
\end{frame}


\begin{frame}
  \frametitle{SessionBasedScript - short recap}
  What the father class will do for you:
  \begin{itemize}
  \item It reads and parses the GC3Pie configuration file.
  \item It creates an \textbf{Engine} class.
  \item It creates a \textbf{Session} to persist jobs.
  \item It parses commonly used command line arguments.
  \item It submit jobs, check their status, fetch their output when
    they are finished.
  \item It automatically sets the following parameters:
    \begin{itemize}
    \item \lstinline|output_dir|
    \item \lstinline|requested_cores|
    \item \lstinline|requested_memory|
    \item \lstinline|requested_walltime|
    \item \lstinline|jobname| 
    \end{itemize}

  \end{itemize}
\end{frame}


\begin{frame}
  \frametitle{SessionBasedScript - customization}
To customize the script you have to modify:
\begin{description}
\item [\texttt{setup\_options(self)}] to add command line options
\item [\texttt{new\_tasks(self, extra)}] method to return a list of
  \textbf{Application}-like instances. Here, you can access command
  line options via \lstinline|self.params.option_name|
\item [\texttt{before\_main\_loop(self)}] to execute some code
  \emph{before} the submission of the jobs.
\item [\texttt{after\_main\_loop(self)}] to execute some code
  \emph{after} the main loop. A list of all Application objects is
  available in the \lstinline|self.session.tasks.values()| list.
\end{description}
\end{frame}


\begin{frame}
  \frametitle{Exercise 5.C} Create a script which will run a variable
  number of copies of the \textbf{CpuinfoApplication} of the
  \lstinline|cpuinfo.py| script and will print the results at the
  end. Remember to:
  \begin{itemize}
  \item in \lstinline|setup_options| add a command line option
    \lstinline|--copies|.
  \item in \lstinline|new_tasks| read the
    \lstinline|self.params.copies| attribute to know how many
    applications to run.
  \item in \lstinline|after_main_loop| check if all the application
    are done, and eventually print the results.
  \end{itemize}
\end{frame}

% problems to solve:
%     exercise A3 must fail:
%     the localhost resource will not allow more than two jobs to run
%         introduce the need for Engine
%         solve exercise A3 properly
%     if the script is killed, information about the jobs is lost
%         introduce the need for a persistence system
%     code duplication: every script must reuse the same scaffolding


% the SessionBasedScript object:
%     SessionBasedScript as a prototypical script with all the options you normally want to manage a set of jobs
%     what parts of the generic HTC script are implemented in the SessionBasedScript
%     define "session": a session is a persisted collection of jobs

%     code example: a session_minimal.py script; run it; look at '--help'; "no jobs in session"

%     how session are constructed and managed:
%     explain the startup sequence of a SessionBasedScript, emphasizing customization points (new_tasks and setup_options particularly)
%     explain the interface of new_tasks and its calling convention

\end{document}

\documentclass[final,english,serif]{beamer}
\usetheme{gc3}

\usepackage[T1]{fontenc}
\usepackage[utf8]{inputenc}
\usepackage[english]{babel}

\usepackage{graphicx}
\usepackage{gc3}

%\title{Py \emph{vs.}~Py}
%\subtitle{A comparison of Python runtimes on a non-numeric scientific code}
\author[R.~Murri et~al.]{Riccardo Murri, Sergio Maffioletti, Antonio Messina}
\institute[GC3]{Grid Computing Competence Center \\ University of Zurich}
\date{EuroSciPy, Aug.~23--24, 2013}

\def\It{\href{http://gc3pie.googlecode.com/}{GC3Pie}\space}

\begin{document}
\begin{frame}
  \frametitle{}
  \begin{center}
    {\Large Yet another remote execution framework? ;-)}

    \\ \+

    \emph{\It is \texttt{subprocess} on steroids}: \\ runs commands on virtual
    machines in the cloud, \\ batch-queueing systems, computational
    grids, \\ or anything you can SSH into.

    \\ \+

    \begin{sloppypar}
      \It can compose tasks into a dynamic workflow, \\ and distribute
      execution  \\ on all available compute power.
    \end{sloppypar}

    \\ \+

    The poster shows how to \\ run \emph{model calibration} on a large
    economic model, \\ requiring $O(10^6)$ CPU hours to compute.
  \end{center}
\end{frame}
\end{document}


%% Local Variables:
%% mode: latex
%% TeX-PDF-mode: t
%% End:
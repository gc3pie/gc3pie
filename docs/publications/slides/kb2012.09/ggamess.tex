%%
%% LaTeX template for UZH presentations.
%% (Riccardo Murri, <riccardo.murri@uzh.ch>)
%% 
%% Requires the `beamer` document class,
%% available at http://latex-beamer.sf.net/
%%
%% Use UTF-8 text (backwards compatible to ASCII, but not to
%% ISO-8859-1 and ISO-8859-15) or change the
%% `\usepackage[utf8x]{inputenc}` line below.
%%

\documentclass[english,serif,mathserif,xcolor=pdftex,dvipsnames,table]{beamer}
\usepackage{lsci}

% \documentclass {beamer}
% \mode<presentation>
% {
%   % for theme/color selection, see: http://www.hartwork.org/beamer-theme-matrix/
%   \usetheme{CambridgeUS}
%   \usecolortheme{dolphin}
  
%   % no navigation bar
%   \setbeamertemplate{navigation symbols}{}

%   \logo{\includegraphics[scale=0.17]{uzh-logo}}
%   % To get just the UniZH coat of arms, use the following line instead:
%   %\logo{\includegraphics[scale=0.25,viewport=0 0 159 159]{uzh-logo}}
% }

% abbrev
\newcommand{\ggamess}{\emph{GGamess}}

\title[GGamess]% will appear on the bottom line
{Running GAMESS jobs \\ with \ggamess}
\author[R.\ Murri]
{Riccardo Murri \\ \texttt{<riccardo.murri@uzh.ch>}}
\institute[GC3, Univ. of Zurich]% will appear on the bottom line
{\href{http://www.gc3.uzh.ch/}{Grid Computing Competence Centre}, 
  \href{http://www.uzh.ch/}{University of Zurich}
  \\ \url{http://www.gc3.uzh.ch/}}
\date[Sep.~11, 2012]% will appear on the bottom line
  {Sep.~11, 2012}


\begin{document}

\subject{Talks}
% This is only inserted into the PDF information catalog. Can be left
% out. 

\begin{frame}
  \titlepage
\end{frame}

% turn logo off after page 1
\expandafter\global\logo{}

% \begin{frame}
%   \frametitle{Talk outline}
%   \tableofcontents
%   % You might wish to add the option [pausesections]
% \end{frame}


%% Consult the `beamer` class documentation to know what kind of LaTeX
%% commands may be used here.  The following is just an example

\section{Introduction}

\begin{frame}
  \frametitle{\ggamess: a tool for high-throughput GAMESS}

  \ggamess\ is a command-line tool to run an arbitrary set of GAMESS
  \texttt{.INP} files in parallel.

  \+ 
  The purpose of this talk is to rehash its usage before we
  introduce the new testing features.

  \+
  If you have any feedback on basic \ggamess, we'd love to hear it as well!
\end{frame}


\section{GGamess}

\begin{frame}
  \frametitle{What is GGamess then?}
  
  GGamess is a command-line tool to submit a set of GAMESS
  \texttt{.INP} files in parallel.
  \begin{itemize}
  \item For each job, manage the entire lifecycle: submit, monitor,
    retrieve output when done.
  \item Each job is independent of others.
  \item You can stop and restart it at a later time, processing
    continues from where it was interrupted.
  \item Finally exits when all jobs are done.
  \end{itemize}
\end{frame}

\begin{frame}[fragile]
  \frametitle{Example: running the GAMESS tests / 1}
  Running \texttt{ggamess} once submits the the jobs.

  \begin{scriptsize}
\begin{semiverbatim}
\$ \textbf{ls tests/data/}
exam01.inp  exam02.inp  exam03.inp  exam04.inp  exam05.inp
\emph{[...]}
exam41.inp  exam42.inp  exam43.inp  exam44.inp 

\$ \textbf{./ggamess.py tests/data/}
Status of jobs in the 'ggamess' session: (at 11:14:23, 05/05/11)
        NEW   0/44     (0.0\%)
    STOPPED   0/44     (0.0\%)
  \textbf{SUBMITTED   44/44   (100.0\%)}
 TERMINATED   0/44     (0.0\%)
TERMINATING   0/44     (0.0\%)
      total   44/44   (100.0\%)
\end{semiverbatim}
  \end{scriptsize}
\end{frame}

\begin{frame}[fragile]
  \frametitle{Example: running the GAMESS tests / 2}
  Running it again updates status and fetches results of finished jobs.

  \+
  You can also request a detailed listing of the jobs with the
  \texttt{-l} option:
  \begin{scriptsize}
    \begin{semiverbatim}
\$ \textbf{./ggamess.py -l test/data/}
\begin{tiny}
 JobID     Job name     State                                    Info                               
===================================================================================
job.8945   exam42     SUBMITTED    Submitted to 'smscg' at Thu May  5 11:11:58 2011                 
job.8944   exam16     SUBMITTED    Submitted to 'smscg' at Thu May  5 11:12:02 2011                 
\emph{[...]}
job.8937   exam20     TERMINATED   Execution of gamess terminated normally
job.8936   exam29     TERMINATED   Execution of gamess terminated normally
\end{tiny}
\end{semiverbatim}
    \end{scriptsize}
    
  Each job has its own output directory.
  \begin{scriptsize}
\begin{semiverbatim}
\$ \textbf{ls exam29}
exam29.dat  exam29.out
\end{semiverbatim}
  \end{scriptsize}
\end{frame}

\begin{frame}[fragile]
  \frametitle{Example: running the GAMESS tests / 3}

  You can tell \texttt{ggamess} to keep running until all jobs are
  done and their output retrieved.

  \+
  Example: keep running, and update job status every 20 seconds.
  \begin{scriptsize}
  \begin{semiverbatim}
\$ \textbf{./ggamess.py tests/data/ -C 20}
Status of jobs in the 'ggamess' session: (at 11:27:48, 05/05/11)
        NEW   0/44     (0.0\%)  
    STOPPED   0/44     (0.0\%)  
  SUBMITTED   5/44    (11.4\%)  
 TERMINATED   39/44   (88.6\%)  
TERMINATING   0/44     (0.0\%)  
         ok   39/44   (88.6\%)  
      total   44/44   (100.0\%) 
\emph{...continues running until Ctrl+C is hit}
\end{semiverbatim}
  \end{scriptsize}
\end{frame}


\begin{frame}[fragile]
  \frametitle{More features / 1}

  Can request a specific version of GAMESS with the \texttt{-R}
  option:
    \begin{itemize}
    \item e.g., \texttt{ggamess -R 2010R3-peve};
    \item default is to use GAMESS 2012R1;
    \item however, still no ``list available versions'' feature;
    \item running customized versions is possible but more complicated
      (ask me if you're interested!)
    \end{itemize}
\end{frame}


\begin{frame}[fragile]
  \frametitle{More features / 2}

  Can run \emph{parallel GAMESS jobs} with the \texttt{-c} option:
  \begin{itemize}
  \item e.g., \texttt{ggamess -c 8} runs a parallel job on 8 CPUs.
  \item the number following \texttt{-c} \emph{must match} what's in
    the \texttt{.INP} file!
  \end{itemize}
\end{frame}


\begin{frame}[fragile]
  \frametitle{More features / 3}

  Option \texttt{-o \emph{directory}} sets a directory under which all
  output will be collected. (Default is the current directory.)

  \+ 
  Option \texttt{-s \emph{name}} collects all job information in a
  \emph{named session}.  You can have several sessions running in
  parallel.
  
  \+ 
  All options can be \emph{independently} combined.
\end{frame}


\section{This is the end...}
\label{sec:end}

\begin{frame}
  \begin{center}
    {\Huge Thank you!}
    \\
    \+
    {\Large (Any questions?)}
  \end{center}
\end{frame}

\begin{frame}
  \frametitle{Three tools for high-throughput GAMESS}

  \begin{description}
  \item[GRunDB] Compute energy of a set of molecules.
  \item[GGamess] Run an arbitrary set of GAMESS \texttt{.INP} files in parallel.
  \item[gc3libs.template] Generate a set of files from a given template.
  \end{description}
\end{frame}


\end{document}

%%% Local Variables: 
%%% mode: latex
%%% TeX-master: t
%%% x-symbol-8bits: nil
%%% End: 
